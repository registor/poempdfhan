\documentclass{ctexart}

\usepackage[a4paper]{geometry}
% 汉风卷轴诗词排版宏包
\usepackage{poempgfhan}

\pagestyle{empty}

% 用\setpinyin{⟨汉字⟩}{⟨拼音⟩}命令全局设置多音字的首选读音,如
% \setpinyin{解}{xie4}
% \setpinyin{冼}{xian3}
% \setpinyin{单}{shan4}

% 也可以用\xpinyin{⟨单个汉字⟩}{⟨拼音⟩}命令对指定的汉字局部设置读音,如
% \xpinyin{单}{dan1},
\begin{document}

\centering

\begin{poem}[1.0][2]{唐}{李白}{将进酒·君不见黄河之水天上来}[-4]
君不见黄河之水天上来,奔流到海不复回。\\
君不见高堂明镜悲白发,朝如青丝暮成雪。\\
人生得意须尽欢,莫使金樽空对月。\\
天生我材必有用,千金散尽还复来。\\
烹羊宰牛且为乐,会须一饮三百杯。\\
岑夫子,丹丘生,将进酒,杯莫停。\\
与君歌一曲,请君为我侧耳听。\\
钟鼓馔玉不足贵,但愿长醉不复醒。\\
古来圣贤皆寂寞,惟有饮者留其名。\\
陈王昔时宴平乐,斗酒十千恣欢谑。\\
主人何为言少钱,径须沽取对君酌。\\
五花马,千金裘,呼儿将出换美酒,与尔同销万古愁。
\end{poem}

\begin{poem}*[1.0][2]{唐}{李白}{\xpinyin{将}{qiang1}进酒·君不见黄河之水天上来}[-4]
君不见黄河之水天上来,奔流到海不复回。\\
君不见高堂明镜悲白发,\xpinyin{朝}{zhao1}如青丝暮成雪。\\
人生得意须尽欢,莫使金樽空对月。\\
天生我材必有用,千金散尽还复来。\\
烹羊宰牛且为乐,会须一饮三百杯。\\
岑夫子,丹丘生,\xpinyin{将}{qiang1}进酒,杯莫停。\\
与君歌一曲,请君为我侧耳听。\\
钟鼓馔玉不足贵,但愿长醉不复醒。\\
古来圣贤皆寂寞,惟有饮者留其名。\\
陈王昔时宴平乐,斗酒十千恣欢谑。\\
主人何为言少钱,径须沽取对君酌。\\
五花马,千金裘,呼儿将出换美酒,与尔同销万古愁。
\end{poem}

  % 默认不加注拼音
  \begin{poem}{现代}{\LaTeX{}er}{赞\textbullet{}\LaTeX{}}
    娟秀轻爽拉泰赫\\
    所写所想即所得\\
    排版何须穷思量\\
    窈窕俊俏尽婀娜
  \end{poem}
  
  % 使用[py]可选参数加注拼音
  \begin{poem}*{现代}{\LaTeX{}er}{赞\textbullet{}\LaTeX{}}
    娟秀轻爽拉泰赫\\
    所写所想即所得\\
    排版何须穷思量\\
    窈窕俊俏尽婀娜
  \end{poem}

\end{document}

%%% Local Variables:
%%% mode: latex
%%% TeX-master: t
%%% End:
